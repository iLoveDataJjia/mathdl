\documentclass[12pt]{article}
\usepackage[OT1]{fontenc}
\usepackage{mickaelpechaud_cours}

\begin{document}

\renewcommand{\labelitemi}{\textbullet}

\date{2008}
  \title{Développements limités}
  \author{Mickaël Péchaud}
  
  \maketitle
  
  \tableofcontents


\newpage
Ce document est sous licence Creative Commons ccpnc2.0 :


\href{http://creativecommons.org/licenses/by-nc/2.0/fr/}{http://creativecommons.org/licenses/by-nc/2.0/fr/} 


En gros, vous pouvez faire ce que bon vous semble avec ce document, y
compris l'utiliser pour faire des papillotes, ou faire une performance
publique (gratuite) durant lequel vous le mangez feuille par feuille (ce
que je déconseille tout de même), aux conditions expresses que :



\begin{itemize}
\item vous en citiez l'auteur.

\item vous n'en fassiez pas d'utilisation commerciale.
\end{itemize}

\bigskip

Par respect pour l'environnement, merci de ne pas imprimer ce document si
ça n'est pas indispensable (et ça n'est pas indispensable).

\newpage

Ce document est un cours introductif sur les développements limités. Il se place à
un niveau MPSI.

\spe{
  Ceci est de niveau MP
}

\;

\spepp{
  Ceci est de niveau au delà de la MP
}


\newpage

\section{Définitions, premières propriétés}

Les développements limités sont un produit de la théorie des comparaisons
de fonctions. L'idée sous-jacente est de comparer toute une classe de
fonctions à une certaine famille de fonctions fixées. Dans le cadre des
développements limités, il s'agira de polynômes. Dit simplement, un
développement limité est donc une approximation en un point d'une fonction
par un polynôme.



% \section{Voisinage}

% \begin{Def}[Voisinage]~

% Soit $I\subset \R$ un intervalle de $R$.

% Soit $a\in I$ ou $I$ une extrémité de $I$. 

% On appelle \defemph{voisinage de $a$} tout intervalle de la forme


%   \begin{itemize}
%   \item $I \cap ]a-\epsilon, a+\epsilon[$
%   \end{itemize}
 
% \end{Def}


\subsection{Développements limités}

\begin{Def}[Développement limité]~

Soit $n\inN$.

Soit $I$ un intervalle de $\R$, $a\in I$.

Soit $f$ une fonction de $I$ dans $\R$. 

On dit que $f$ admet un \defemph{développement limité} (DL) à l'ordre $n$
en $a$ ($\DL_n(a)$) Ssi il existe $P \in \R_n[X]$ et $\epsilon :
I\rightarrow \R$ tels que pour tout $x\in I$

\begin{center}

 $$f(x)= P(x-a) + (x-a)^{n}\epsilon(x)$$ et 
 $$\lim_a \epsilon=0$$

\end{center}




$x\mapsto P(x-a)$ est appelée \defemph{partie régulière} du DL,
$x\mapsto (x-a)^{n}\epsilon(x)$ est appelé \defemph{reste} d'ordre $n$.

On peut également écrire : 

$f(x)\underset{x\rightarrow a}{=} P(x-a) + \text{o}(x-a)^{n}$

\end{Def}

Un développement limité est l'approximation \textbf{locale} d'une fonction
par un polynôme. L'approximation est d'autant plus précise que l'ordre du
DL est élevé.


\begin{Exems}~
  \begin{itemize}

  \item Soit $f:x\mapsto x^3-2x^2+3x+1$. Pour tout $x$, on a $f(x)=1+3x-2x^2+x (x^2)$, avec $x^2
    \underset{x\rightarrow 0}{\rightarrow} 0 $.

    Donc $f$ admet un $\DL_2(0)$, de partie régulière $x\mapsto 1+3x-2x^2$.
    
    $f$ admet de même un  $\DL_3(0)$, de partie régulière $x\mapsto 1+3x-2x^2+x^3$,
    et de reste nul. En fait, il s'agit également d'un $\DL_n(0)$.
    
    Plus généralement, on a la propriété suivante :
    
    \begin{Prop}~      
      $$ f \text{admet un } \DL_n(a) \text{de reste nul}
      \Leftrightarrow f \text{est un polynôme de degré au plus\;} n $$
    \end{Prop}
    \begin{demo}~
      
      
      \begin{itemize}
        \item pour le sens direct, on écrit simplement la définition du DL
        pour un tel ordre $n$. Le reste étant nul, la fonction est égale à
        sa partie régulière, qui est un polynôme de degré au plus $n$.
        
        \item pour le sens réciproque, on pose $g(.)=f(.+a)$, et on écrit
        pour tout $x\inR$ $f(x)=g(x-a)=g(x-a)+\text{o}((x-a)^n)$, qui est un $\DL_n(a)$ de reste
        nul.
      \end{itemize}


    \end{demo}

    Les DLs ne sont donc pas un outil très intéressant pour étudier les
    polynômes\dots
    
  \item Soit $f$ définie par

\[
f : \left\{
\begin{array}{rcl}
]-1,1[ & \to & \R \\
x & \mapsto & \frac{1}{1-x}
\end{array}
\right.
\]

On montre par récurrence que pour tout $n \inN$ et pour tout $x\inR\backslash\{1\}$, $$\frac{1}{1-x} = 1 + x +
\dots + x^n + x^n \left(\frac{x}{1-x}\right)$$

$x\mapsto \frac{x}{1-x}$ étant de limite nulle en $0$, $f$ admet donc un
$\DL_n(0)$, de partie régulière $ x\mapsto 1 + x + \dots + x^n$.

  \end{itemize}
\end{Exems}

\paragraph{Remarque:}$f$ admet un $\DL_n(a)$ Ssi $f(.-a)$ admet un
$\DL_n(0)$. On pourra donc faire l'étude des propriétés théoriques des DLs
en $0$.

\subsection{Troncature}

Si on a un DL a un certain ordre $n$, on obtient facilement un DL d'ordre
$p<n$ en tronquant la partie régulière du $\DL_n(a)$ à l'ordre $p$.

\begin{Prop}~

Si $f$ admet un $\DL_n(a)$, alors $f$ admet un $\DL_p(a)$ pour tout $p<n$.

Sa partie régulière est celle du $\DL_n(a)$ tronquée à l'ordre $p$.


\end{Prop}


\begin{demo}~
  
Sans perte de généralité, étudions le cas $a=0$.

$f(x)\underset{x\rightarrow 0}{=} a_0 + x a_1 + \dots + x^p a_p + x^{p+1} a_{p+1} + \dots + x^n a_n  + \text{o}(x)^{n}$

Or, $x^{p+1} a_{p+1} + \dots + x^n a_n  + \text{o}(x)^{n}
\underset{x\rightarrow 0}{=} \text{o}(x)^{p}$

D'où $f(x) \underset{x\rightarrow 0}{=} a_0 + x a_1 + \dots + x^p a_p + \text{o}(x)^{p}$


\end{demo}



\subsection{Unicité}

Bonne nouvelle, il y a unicité du développement limité.

\begin{Theo}[Unicité du DL]~

Soit $f:I\rightarrow \R$, $a\in I$, $n\inN$. Il y a unicité de la partie
régulière du $\DL_n(a)$.

\end{Theo}

\begin{demo}~

Sans perte de généralité, traitons le cas $a=0$.

Supposons l'existence de deux polynômes $P$ et $Q \in \R_n[X]$ tels que 

$f(x)\underset{x\rightarrow 0}{=} P(x) + \text{o}(x)^{n}$

et 

$f(x)\underset{x\rightarrow 0}{=} Q(x) + \text{o}(x)^{n}$

On en déduit donc que $(P-Q) (x) \underset{x\rightarrow 0}{=} \text{o}(x)^{n}$. Or $\deg(P-Q)\leq n$. On
en déduit immédiatement $P-Q=0$.

\end{demo}


\begin{Cor}~


\begin{itemize}
\item Soit $f$ une fonction paire. La partie régulière du $\DL_n(0)$ est un
polynôme pair.

\item Soit $f$ une fonction impaire. La partie régulière du $\DL_n(0)$ est un
polynôme impair.
\end{itemize}

\end{Cor}

\begin{demo}~

On écrit $f(x)\underset{x\rightarrow 0}{=}P(x)+\text{o}(x^n)$. Par parité de $f$, on obtient
$P(-x)+\text{o}(x^n)\underset{x\rightarrow 0}{=}f(-x)\underset{x\rightarrow 0}{=}f(x)\underset{x\rightarrow 0}{=}P(x)+\text{o}(x^n)$. Par unicité du DL, on
en déduit $P(X)=P(-X)$.

On raisonne de même pour les fonctions impaires.

\end{demo}

\subsection{Liens avec la continuité et la dérivabilité}

À l'ordre 0, nous avons une caractérisation très simple pour l'existence de
DLs. 

\begin{Prop}~

$f$ admet un $\DL_0(a)$ Ssi $f$ est continue en $a$.

\end{Prop}

En effet, la définition se réecrit $f(x)\underset{x\rightarrow 0}{=}a_0+\text{o}(1)$. En passant à la
limite, on a $a_0=f(a)$. On a donc $f(x)\underset{x\rightarrow
  0}{=}f(a)+\text{o}(1)$, ce qui est exactement
la définition de la continuité en $a$.

\medskip

À l'ordre 1, cela va encore. Comme ci-dessus, on peut écrire :

\begin{Prop}~

$f$ admet un $\DL_1(a)$ Ssi $f$ est dérivable en $a$.

\end{Prop}

et on montre que la partie régulière du DL est $x\mapsto f(a)+f'(a)(x-a)$.


\medskip

Dans la section suivante, nous allons partiellement généraliser ces
résultats aux ordres supérieurs. 

\section{Taylor-Young}

\subsection{Le théorème}

Voici maintenant un théorème qui assure l'existence de DLs à des ordres
supérieurs pour une grande classe de fonctions, et permet de plus de les
calculer simplement.

\begin{Theo}[Taylor-Young]~ 

\label{TaylorYoung}

Soit $n \inN^*$, $I$ un intervalle de $\mathbb{R}$, $a$ dans l'intérieur de
$I$. Soit $f$ une fonction de $I$ dans $\mathbb{R}$, $n$ fois dérivable
dans $I$.

Alors, $f$ admet un $\DL_n(a)$, de partie réulière
$\sum_{k=0}^n\frac{f^{(k)}(a)}{k!}(X-a)^k$, i.e.

$$f(x)\underset{x\rightarrow a}{=}\sum_{k=0}^n\frac{f^{(k)}(a)}{k!}(x-a)^k+\text{o}((x-a)^n)$$

\end{Theo}

\begin{demo}~

Effectuons la démonstration par récurrence.
  
\begin{description}
\item[Initialisation:] si $n=1$, $f$ est dérivable, et on a par définition
  $f'(a)=\underset{x\rightarrow a}{\lim}\frac{f(x)-f(a)}{x-a}$, i.e. $f(x)-f(a)\underset{x\rightarrow a}{\sim} f'(a)(x-a)$ ou
  encore $f(x)-f(a) \underset{x\rightarrow a}{=} f'(a)(x-a)+\text{o}((x-a)^1)$

\item[Hérédité:] supposons le résultat vrai pour $n$. Soit $f$ $n+1$ fois dérivable
  dans $I$. Alors $f'$ est $n$ fois dérivable. On lui applique l'hypothèse
  de récurrence, et on peut écrire :

$$f'(x)-\sum_{k=0}^n\frac{f^{(k+1)}(a)}{k!}(x-a)^k\underset{x\rightarrow a}{=}\text{o}(x-a)^n$$

Pour tout $\ep>0$ il existe donc $\eta$ tel que $$\forall x\in I\;\;|x-a|\leq\eta \Rightarrow
\left|f'(x)-\sum_{k=0}^n\frac{f^{(k+1)}(a)}{k!}(x-a)^k\right| \leq \ep |x-a|^n$$

Nous voudrions primitiver cette inégalité, mais les valeurs absolues sont
génantes. Distinguons deux cas :

\begin{itemize}

\item Soit $x\in [a, a+\eta]$. On a $$-\ep (x-a)^n \leq f'(x)-\sum_{k=0}^n\frac{f^{(k+1)}(a)}{k!}(x-a)^k \leq \ep (x-a)^n$$

Primitivons entre $a$ et $x$. D'après l'inégalité des accroissements finis, on a 


$$-\frac{\ep}{n+1}(x-a)^{n+1}\leq f(x)-\sum_{k=0}^n\frac{f^{(k+1)}(a)}{(k+1)!}(x-a)^{k+1}-f(a)\leq\frac{\ep}{n+1} (x-a)^{n+1}$$

Soit 

$$\forall x\in [a, a+\eta]\;\;\;\;\left|f(x)-\sum_{k=0}^{n+1}\frac{f^{(k)}(a)}{k!}(x-a)^{k}\right|\leq\frac{\ep}{n+1} |x-a|^{n+1}$$

\item Soit $x\in [a-\eta, a]$. On a $$-\ep (a-x)^n \leq f'(x)-\sum_{k=0}^n\frac{f^{(k+1)}(a)}{k!}(x-a)^k \leq \ep (a-x)^n$$

et on conclut de la même manière.

\end{itemize}

On obtient donc finalement 

$$\forall x\in [a-\eta, a+\eta]\;\;\;\;\left|f(x)-\sum_{k=0}^{n+1}\frac{f^{(k)}(a)}{k!}(x-a)^{k}\right|\leq\frac{\ep}{n+1} |x-a|^{n+1}$$

et quitte à poser $\ep'=\frac{\ep}{n+1}$, on a donc bien le résultat voulu.

\end{description}

\end{demo}

En revanche, la réciproque est fausse. À partir de l'ordre $n=2$, on peut
trouver des fonctions admettant des $\DL_n$, mais qui ne sont pas $n$ fois
dérivables.

\begin{Exem}~

Considérons

\[
f : \left\{
\begin{array}{rcl}
\R  & \to & \R \\
x & \mapsto & \left\{
\begin{array}{cl} 
  x^3 \sin(\frac{1}{x}) & \text{si}\;\; x\neq 0\\
  0 & \text{si}\;\; x=0 
\end{array}
\right.
\end{array}
\right.
\]

$f$ est $\cinf$ sur $\R^*$

Un calcul rapide montre que pour $x\inR^*$, 

$$f'(x)=3x^2\sin(\frac{1}{x}) - x
\cos(\frac{1}{x})$$

qui est de limite nulle quand $x$ tend vers 0.

Par ailleurs, en considérant le taux d'accroissement en 0, on voit
facilement que $f$ est dérivable en $0$ et que $f'(0)=0$ -- d'où l'on déduit
que $f \in \mathcal{C}^1$.

Analysons la dérivabilité de $f'$ en $0$ :

$$\frac{f'(x) - f'(0)}{x-0}=3x\sin(\frac{1}{x}) - 
\cos(\frac{1}{x})$$

qui n'a pas de limite lorsque $x$ tend vers 0.

D'où $f
\notin \Delta^2$.

 
En revanche, en posant $g:x\mapsto x\sin(\frac{1}{x})$ prolongée pas $0$ en
$0$, on peut écrire pour tout $x\inR^*$ $f(x)=0+0x+0x^2+x^2(g(x))$, qui
est un $\DL_2(0)$ de $f$. 



\end{Exem}


\subsection{Calculs de DLs}

Ce théorème nous permet de calculer immédiatement des DLs pour de
nombreuses fonctions usuelles.

\subsubsection{Exponentielle}

$\exp$ est $\cinf$, et sa dérivée n\textsuperscript{ème} est
elle-même. D'après Taylor-Young (\ref{TaylorYoung}), elle admet un DL à
tout ordre $n$ en 0 :

\begin{align}
\exp(x)&\underset{x\rightarrow 0}{=}\sum_{k=0}^n \frac{x^k}{k!}+\text{o}(x^n)\\
&\underset{x\rightarrow 0}{=}1+x+\frac{x^2}{2!}+\frac{x^3}{3!}+\dots
\end{align}


Par définition de $\sh$ et $\ch$ (comme partie impaires et paires de
$\exp$), on obtient immédiatement leur DL respectif :

\begin{align}
  \ch(x)=\frac{e^x+e^{-x}}{2}&\underset{x\rightarrow 0}{=}\sum_{k=0}^{n/2} \frac{x^{2k}}{(2k)!}+\text{o}(x^n)\\
  &\underset{x\rightarrow 0}{=}1+\frac{x^2}{2!}+\frac{x^4}{4!}+\dots
\end{align}

\begin{align}
  \sh(x)=\frac{e^x-e^{-x}}{2}&\underset{x\rightarrow 0}{=}\sum_{k=0}^{(n-1)/2} \frac{x^{2k+1}}{(2k+1)!}+\text{o}(x^n)\\
  &\underset{x\rightarrow 0}{=}x+\frac{x^3}{3!}+\frac{x^5}{5!}+\dots
\end{align}


Remarquez que la parité de $\ch$ entraine la nullité des termes impairs,
et nous permet donc de gagner un ordre à la fin : ainsi, on a par exemple 

$$  \ch(x) \underset{x\rightarrow 0}{=}1+\frac{x^2}{2!}+\frac{x^4}{4!}+\text{o}(x^{\mathbf{5}})\dots
$$

et de même pour $\sh$.


\subsubsection{Fonctions trigonométriques}

$\cos$ et $\sin$ sont $\cinf$, et admettent donc un développement limité à
tout ordre en $0$.


Par ailleurs, on a $\cos'=-\sin$, $\cos^{(2)}=-\cos$, $\cos^{(3)}=\sin$,
$\cos^{(4)}=\cos$, et ainsi de suite. 

Les valeurs des dérivées successives
en $0$ sont donc $1$, $0$, $-1$, $0$, $1$, $0$, $-1$, $0$\dots 
 

On en déduit simplement la partie régulière du DL$_{2n+1,0}$ de $\cos$ :

\begin{align}
\cos(x)&\underset{x\rightarrow 0}{=}\sum_{k=0}^{n} (-1)^k\frac{x^{2k}}{(2k)!}+\text{o}(x^{2n+1})\\
&\underset{x\rightarrow 0}{=}1-\frac{x^2}{2!}+\frac{x^4}{4!}+\dots
\end{align}



De même, pour $\sin$, 

\begin{align}
\sin(x)&\underset{x\rightarrow 0}{=}\sum_{k=0}^{n} (-1)^k\frac{x^{2k+1}}{(2k+1)!}+\text{o}(x^{2n+2})\\
&\underset{x\rightarrow 0}{=}x-\frac{x^3}{3!}+\frac{x^5}{5!}+\dots
\end{align}

Ici aussi, la nullité d'un terme sur deux permet de gagner un ordre à la
fin.



\subsubsection{Puissances}

Soit $\alpha \inR$.

Considérons la fonction $f:x\mapsto(1+x)^\alpha$. Elle est $\cinf$ en $0$, et on
calcule sans peine ses dérivées pour $x$ au voisinage de $0$ :
$f'(x)=\alpha(1+x)^{\alpha-1}$,
$f''(x)=\alpha(\alpha-1)(1+x)^{\alpha-2}$\dots \;(les dérivées finissent par s'annuler si $\alpha\inN$, i.e. si $f$ est un polynôme).

On a donc

$$(1+x)^\alpha \underset{x\rightarrow 0}{=} 1+\alpha x + \frac{\alpha(\alpha-1)}{2!} x^2 +
\frac{\alpha(\alpha-1)(\alpha-2)}{3!} x^3 + \dots$$

Quelques cas particuliers fréquents : pour tout $n$, on a :

\begin{description}

\item[$\alpha=1/2$]

$$\sqrt{1+x} \underset{x\rightarrow 0}{=} 1 +\frac{1}{2} x - \frac{1}{8}
  x^2 - (-1)^n \frac{1.3.5\dots(2n-3)}{2.4\dots(2n)} x^n+  \text{o}(x^n)$$

\item[$\alpha=-1/2$]

$$\frac{1}{\sqrt{1+x}} \underset{x\rightarrow 0}{=} 1 -\frac{1}{2} x + \frac{3}{8} x^2 + (-1)^n \frac{1.3.5\dots(2n-1)}{2.4\dots(2n)} x^n+ \text{o}(x^n)$$

\item[$\alpha=-1$]

$$\frac{1}{1+x} \underset{x\rightarrow 0}{=} 1 - x +  x^2 - x^3 + x^4 +
  \dots + (-1)^nx^n + \text{o}(x^n)$$

\end{description}


\section{Opérations sur les DLs}

Voyons mainenant comment combiner ces formules.

Dans toute cette section, $n$ est un entier quelconque. 

\subsection{Combinaison linéaire}

\begin{Prop}~

Soient $f$ et $g$ admettant un $\DL_n(a)$, et $(\lambda,
\mu)\inR^2$. Soient $P$ et $Q$ les parties régulières de ces DLs. Alors
$\lambda f + \mu g$ admet un $\DL_n(a)$ de partie régulière $\lambda P +
\mu Q$.

\end{Prop}

\begin{demo}~

Soient $f$ et $g$ comme dans l'énoncé.

On a $$f(x)\underset{x\rightarrow a}{=}P(x-a)+\text{o}((x-a)^n)$$ et
$$g(x)\underset{x\rightarrow a}{=}Q(x-a)+\text{o}((x-a)^n)$$ 

On en déduit 
$$(\lambda f+\mu g)(x)\underset{x\rightarrow a}{=}(\lambda P+\mu Q)(x-a)+\text{o}((x-a)^n)$$

\end{demo}


\subsection{Produit}

\begin{Prop}~

Soient $f$ et $g$ admettant un $\DL_n(a)$. Soient $P$ et $Q$ les parties
régulières de ces DLs. Alors $f g$ admet un $\DL_n(a)$. Sa partie régulière
est obtenue comme troncature à l'ordre $n$ de $PQ$.

\end{Prop}

\begin{demo}~

 Plaçons-nous en $a=0$ sans perte de
généralité. 
Soient $f$ et $g$ comme dans l'énoncé.

On a $$f(x)\underset{x\rightarrow 0}{=}P(x)+\text{o}(x^n)$$ et
$$g(x)\underset{x\rightarrow 0}{=}Q(x)+\text{o}(x^n)$$ 

On en déduit par produit
$$(fg)(x)\underset{x\rightarrow 0}{=}(PQ)(x)+\text{o}(x^n)$$

Notons $\widetilde{PQ}$ la troncature à l'ordre $n$ de $PQ$. On a
$\widetilde{PQ}(x)\underset{x\rightarrow 0}{=}PQ(x)+\text{o}(x^n)$ d'où 

$$(fg)(x)\underset{x\rightarrow 0}{=}(\widetilde{PQ})(x)+\text{o}(x^n)$$

De plus, $d^o(\widetilde{PQ}) \leq n$, et $\widetilde{PQ}$ est donc bien la partie
régulière du $\DL_n(0)$ de $fg$.

\end{demo}

\begin{Exem}~

Soit à obtenir un $\DL_4(0)$ de $\cos\sin$.

On écrit 

$$\cos(x)\underset{x\rightarrow 0}{=}1-\frac{x^2}{2}+\frac{x^4}{4!}+\text{o}(x^4)$$

$$\sin(x)\underset{x\rightarrow 0}{=}x-\frac{x^3}{3!}+\text{o}(x^4)$$

Le produit des parties régulières donne :

$$(1-\frac{x^2}{2}+\frac{x^4}{4!})(x-\frac{x^3}{3!}) = x - \frac{2x^3}{3}   + \frac{x^5}{8}  - \frac{ x^7}{144}
\underset{x\rightarrow 0}{=} x-\frac{2x^3}{3} + \text{o}(x^4)$$ 

D'où $$\cos(x)\sin(x)\underset{x\rightarrow 0}{=}x-\frac{2x^3}{3} + \text{o}(x^4)$$ 

\end{Exem}

\begin{Exem}~

Voici une autre méthode pour obtenir un $\DL_4(0)$ de $\cos\sin$.

On écrit pour tout $x$ réel $\cos(x)\sin(x)=\frac{\sin(2x)}{2}$.

On sait de plus que $\sin(x)\underset{x\rightarrow 0}{=}x-\frac{x^3}{3!}+\text{o}(x^4)$. 

Par composition à droite, on a donc $\sin(2x)\underset{x\rightarrow
  0}{=}2x-\frac{8 x^3}{3!}+\text{o}(x^4)$. 

D'où 

$$\cos(x)\sin(x)\underset{x\rightarrow
  0}{=}x-\frac{2 x^3}{3}+\text{o}(x^4)$$

\end{Exem}


\subsection{Composition}

\begin{Prop}~

\label{compoDL}

Soient $f$ admettant un $\DL_n(a)$  et $g$ admettant un
$\DL_n(b)$. Supposons de plus $g(b)=a$. Soient $P$ et $Q$ les parties
régulières de ces DLs. Alors $f\text{o} g$ admet un $\DL_n(a)$. Sa partie régulière
est obtenue comme troncature à l'ordre $n$ de $P \text{o} Q$.

\end{Prop}

\begin{demo}~

Quitte à effectuer un changement de variable et à poser $g\rightarrow
g-g(a)$, on peut sans perte de généralité se ramener au cas où $a=b=0$.

On a pour tout $x$ sur un voisinage de $0$ $$f(x)=P(x)+x^n\ep_f(x)$$ et
$$g(x)=Q(x)+x^n\ep_g(x)$$ 

avec $\ep_f$ et $\ep_g$ de limite nulle en 0.

Écrivons $P(X)=a_n X^n + \dots +a_0$.

On a alors

\begin{align*}
f\text{o}g(x)&=a_n g(x)^n+ \dots + a_1 g(x) + a_0 +x^n\ep_f(x)\\
&=a_n (Q(x)+x^n \ep_g(x))^n+ \dots + a_1 (Q(x)+x^n \ep_g(x)) + a_0 +x^n\ep_f(x)
\end{align*}

Or, pour tout $i\in [0,n]$, $$(Q(x)+x^n \ep_g(x))^i=\sum_{k=0}^i
\binom{k}{i}Q(x)^{i-k} (x^n \ep_g(x))^{k}=Q(x)^i + x^n \sum_{k=1}^i
\left(\binom{k}{i}Q(x)^{i-k} x^{n(k-1)}\right) \ep_g(x)^{k} $$

Chacun des termes de la somme tend vers $0$ lorsque $x$ tend vers $0$.

On en déduit $$(Q(x)+x^n \ep_g(x))^i=Q(x)^i+x^n\ep_i(x)$$

où $\ep_i$ est de limite nulle en 0.

En réinjectant dans l'expression de $f\text{o} g$, on obtient

\begin{align*}
f\text{o}g(x)&=a_n Q(x)^n+ \dots + a_1 Q(x) + a_0
+x^n(\ep_f(x)+\ep_1(x)+\dots+\ep_n(x) )\\
&=P\text{o}Q(x)+x^n(\ep_f(x)+\ep_1(x)+\dots+\ep_n(x) )\\
\end{align*}

Par troncature, on en déduit le résultat annoncé.

\end{demo}




\subsection{Inverse}

\begin{Prop}~

Soient $f$ admettant un $\DL_n(a)$ et telle que $f(a)\neq 0$. Alors $\frac{1}{f}$ admet un $\DL_n(a)$. 
\end{Prop}

\begin{demo}~

Soit $n\inN$, et $f$ vérifiant les hypothèses de l'énoncé.

Posons $$g:x\mapsto \frac{1}{x+f(a)}$$ $g$ admet un $\DL_n(0)$.

On a alors pour tout $x$ au voisinage de $a$
$$\frac{1}{f(x)}=\frac{1}{f(x)-f(a)+f(a)} = g\text{o}(f-f(a))(x)$$ 

D'après \ref{compoDL}, $\frac{1}{f(x)}$ admet donc un $\DL_n(a)$.

\end{demo}

La démonstration nous fournit également le moyen de calculer ce
développement limité.

\begin{Exem}~

Voici un exemple de calcul pratique :

Soit $f:x\mapsto \frac{1}{cos(x)}$

Cherchons un $\DL_4(0)$ de $f$. D'après la proposition, comme $cos(0)=1$,
un tel DL existe.

Écrivons pour tout $x$ au voisinage de 0 

$$f(x)=\frac{1}{\cos(x)-1+1}$$

On a $$1-\cos(x)\underset{x\rightarrow
  0}{=}\frac{x^2}{2}-\frac{x^4}{4!}+\text{o}(x^4)$$

et $$\frac{1}{1-x}\underset{x\rightarrow
  0}{=}1+x+x^2+x^3+x^4+\text{o}(x^4)$$

D'après \ref{compoDL}, la partie régulière recherchée est la composée de
ces parties régulières tronquée à l'ordre $4$, i.e.

$$f(x)\underset{x\rightarrow
  0}{=}1+\left(\frac{x^2}{2}-\frac{x^4}{4!}\right)+\left(\frac{x^2}{2}\right)^2+\text{o}(x^4)\underset{x\rightarrow
  0}{=}1+\frac{x^2}{2}+\frac{5x^4}{4!}+\text{o}(x^4)$$ 


On peut vérifier la cohérence de ce résultat effectuant le produit avec le
$\DL_4(0)$ de $\cos$.

\end{Exem}



\subsection{Primitivation de DLs}

\begin{Prop}~

\label{primiDL}
  Soit $f:I\rightarrow \R$ dérivable. Si $f'$ admet un $\DL_n(a)$, de partie
  régulière $x\rightarrow a_0+a_1x+\dots a_nx^n$, alors $f$ admet un $\DL_{n+1}(a)$ de partie régulière $x\rightarrow f(a)+a_0x+a_1\frac{x^2}{2}+\dots a_n\frac{x^{n+1}}{n+1}$.

  
\end{Prop}

En d'autres termes, on peut primitiver terme à terme un DL.

\begin{demo}~

Il suffit de reprendre la preuve de Taylor-Young (\ref{TaylorYoung}).
  
\end{demo}

Ce théorème permet de calculer facilement certains DLs.


\subsubsection{Logarithme}

Nous avons déjà vu que pour tout $n$, 

$$\frac{1}{1-x} \underset{x\rightarrow 0}{=} 1 + x +  x^2 + x^3 + \dots + x^n +
\text{o}(x^n)$$

Par primitivation de DL, on en déduit immédiatement que 

$$\ln(1-x) \underset{x\rightarrow 0}{=} x + \frac{x^2}{2} + \frac{x^3}{3}
\dots + \frac{x^n}{n} + \text{o}(x^n)$$


\subsubsection{Arctangente}

On a $\arctan'= x\rightarrow \frac{1}{1+x^2}$.

Or par composée de DLs, 

$$\frac{1}{1+x^2}\underset{x\rightarrow 0}{=} 1-x^2+x^4-x^6+x^8 \dots +(-1)^{n}x^{2n}+\text{o}(x^{2n+1})$$

Par primitivation de DLs, on obtient 

$$\arctan(x)\underset{x\rightarrow 0}{=}
x-\frac{x^3}{3}+\frac{x^{5}}{5}-\frac{x^{7}}{7}+ \dots +
(-1)^{n}\frac{x^{2n+1}}{2n+1}+\text{o}(x^{2n+2}) $$


\subsubsection{Fonctions circulaires réciproques}

On a $\arcsin'= x\rightarrow \frac{1}{\sqrt{1-x^2}}$.

De même que précédemment, on obtient pour tout $n$

$$\frac{1}{\sqrt{1-x^2}} \underset{x\rightarrow 0}{=} 1 +\frac{1}{2} x^2 + \frac{3}{8} x^4 + 1 \frac{1.3.5\dots(2n-1)}{2.4\dots(2n)} x^{2n}+ \text{o}(x^{2n+1})$$

d'où en primitivant 

$$\arcsin(x) \underset{x\rightarrow 0}{=} x +\frac{1}{2.3} x^3 +
\frac{3}{8.5} x^5 +  \frac{1.3.5\dots(2n-1)}{2.4\dots(2n)} \frac{x^{2n+1}}{(2n+1)} + \text{o}(x^{2n+2})$$

Le $\DL_n(0)$ de $\arccos$ s'obtient immédiatement en remarquant que
$\arccos+\arcsin=\pi/2$ sur $[-1,1]$ :

$$\arccos(x) \underset{x\rightarrow 0}{=} \pi/2 - x -\frac{1}{2.3} x^3 -
\frac{3}{8.5} x^5 -  \frac{1.3.5\dots(2n-1)}{2.4\dots(2n)} \frac{x^{2n+1}}{(2n+1)} + \text{o}(x^{2n+2})$$


\subsection{Dérivation de DLs}

Pour les dérivations de DLs, ça ne se passe pas aussi bien.

\begin{Prop}~

  Soit $f:I\rightarrow \R$ dérivable. Si $f$ admet un $\DL_{n+1}(a)$ de
  partie régulière $P$ et si $f'$ admet un $\DL_n(a)$ de partie régulière
  $Q$, alors $P'=Q$.
  
\end{Prop}

\begin{demo}~

C'est un corrolaire immédiat de \ref{primiDL}.

\end{demo}


Ce qui nous dit que l'on peut dériver des Dls d'ordre $n+1$
  \textbf{uniquement si l'on sait que la dérivée admet un DL d'ordre
  $n$}. Repenser à notre exemple :

\[
f : \left\{
\begin{array}{rcl}
\R  & \to & \R \\
x & \mapsto & \left\{
\begin{array}{cl} 
  x^3 \sin(\frac{1}{x}) & \text{si}\;\; x\neq 0\\
  0 & \text{si}\;\; x=0 
\end{array}
\right.
\end{array}
\right.
\]

qui admet un $\DL_2(0)$, mais n'est pas $\Delta^2$.

$f'$ n'est pas donc pas dérivable, et n'admet donc pas de $\DL_1(0)$.

\section{Aller jusqu'à l'infini ?}

Une question que l'on peut naturellement se poser est \og que se passe-t-il
si on effectue un DL jusqu'à l'infini \fg, et en particulier \og la partie
régulière d'un DL tend-elle vers la fonction lorsque l'ordre du DL tend
vers l'infini \fg.

La réponse à cette question est non. Comprendre dans quel cadre ces
assertions sont vraies relève de la théorie des séries entières, donc
contentons-nous ici d'un seul exemple.

Soit $f$ définie par

\[
f : \left\{
\begin{array}{rcl}
\R  & \to & \R \\
x & \mapsto & \left\{
\begin{array}{cl} 
  \exp(-\frac{1}{x}) & \text{si}\;\; x > 0\\
  0 & \text{si}\;\; x\leq 0 
\end{array}
\right.
\end{array}
\right.
\]

Montrons que pour tout $n$ $f(x)\underset{x\rightarrow 0}{=}
\text{o}(x^n)$. 

Considérons pour tout $x\neq 0$ le rapport $f(x)/x^n$. 

\begin{itemize}

\item Si $x<0$, ce rapport est nul. 

\item Si $x>0$, effectuons le changement de variable $t=1/x$. On obtient
  $f(x)/x^n = t^n exp(-t)$. Par croissance comparée des puissances et de
  l'exponentielle en $+\infty$, on a donc $\lim_{x\rightarrow
  0}f(x)/x^n=0$.
 
\end{itemize}

On en déduit que $f$ admet un $\DL_n(0)$ à tout ordre $n$ :

$$f(x)\underset{x\rightarrow 0}{=} 0+0x+0x^2+\dots+0x^n+\text{o}(x^n)$$

Le DL à tout ordre de $n$ est donc nul ! Il n'est donc pas question que le
DL \og converge \fg\; vers la fonction pour tout réel positif.

\section{Calculs pratiques de DLs}

\begin{itemize}

\item Calculons un $\DL_4(1)$ de $f:x\rightarrow \frac{1}{\exp(x)}$.

Pas besoin ici de sortir des calculs compliqués. On a pour tout $x\inR$
$f(x)=\exp(-x)$, et Taylor-Young nous donne immédiatement le résultat :

$$f(x)\underset{x\rightarrow 1}{=} \frac{1}{e} - \frac{(x-1)}{e} +
\frac{(x-1)^2}{2! e}
- \frac{(x-1)^3}{3! e} +  \frac{(x-1)^4}{4! e} + \text{o}((x-1)^4) $$

\textbf{Pas la peine d'aller faire des calculs de DLs pénibles si les
  dérivées sont triviales à calculer\dots}

\item Calculons un $\DL_3(1)$ de $f:x\rightarrow \frac{1}{\exp(x)+x}$.

Les dérivées successives sont ici un peu plus pénibles à calculer. C'est
parti pour un calcul de $\DL$.

Les formules que nous connaissons pour les DLs sont en $0$. Commençons par
effectuer un changement de variable $x \leftrightarrow t+1$.

On obtient $$\frac{1}{\exp(x)+x}=\frac{1}{\exp(t+1)+t+1} $$ 

À l'ordre 3 en 0, on a 

$$\exp(t+1)+t+1=e\exp(t)+t+1\underset{t\rightarrow o}{=} e+1 +(e+1)t
+\frac{e}{2} t^2 + \frac{e}{3!} t^3 + \text{o}(t^3) $$

Pour calculer le DL d'un quotient, on cherche à se ramener à une
composition avec $x\rightarrow \frac{1}{1\pm x}$ avec $x\rightarrow 0$.

Il va falloir visiblement factoriser par $e+1$ pour arriver à quelquechose
de cette forme.

Écrivons avec un abus de notation\footnote{l'abus de notation consiste à
  écrire des $\text{o}$ sous le trait de fraction, ou plus généralement à
  l'intérieur d'une formule. Il faut penser le $\text{o}$ comme une
  fonction négligeable devant la quantité indiquée.}

\begin{align*}
\frac{e+1}{\exp(t+1)+t+1} &\underset{t\rightarrow 0}{=}  \frac{1}{1+t +\frac{e}{2(e+1)} t^2
  + \frac{e}{3!(e+1)} t^3 + \text{o}(t^3)}\\
&= 1- (t +\frac{e}{2(e+1)} t^2 + \frac{e}{3!(e+1)} t^3 +
  \text{o}(t^3)) \\&+  (t +\frac{e}{2(e+1)} t^2
  + \frac{e}{3!(e+1)} t^3 + \text{o}(t^3))^2 \\&- (t +\frac{e}{2(e+1)} t^2
  + \frac{e}{3!(e+1)} t^3 + \text{o}(t^3))^3 + \text{o}(t^3)
\end{align*}

En développant et tronquant, on obtient

\begin{align*}
\frac{1+e}{\exp(t+1)+t+1} &\underset{t\rightarrow 0}{=} 1 -
 t + 1/2 \frac{e+2}{e+1} t^2 - 1/6 \frac{e+6}{e+1} t^3 + \text{o}(t^3)
\end{align*}

soit, en revenant à la variable de départ :

\begin{align*}
f(x) &\underset{x\rightarrow 1}{=} \frac{1}{e+1} - \frac{1}{e+1}
 (x-1) + 1/2 \frac{e+2}{(e+1)^2} (x-1)^2 - 1/6 \frac{e+6}{(e+1)^2} (x-1)^3 + \text{o}((x-1)^3)
\end{align*}



\end{itemize}



\section{Applications}



\subsection{Calculs de limites/équivalents}

Quand l'utilisation d'équivalent ne suffit pas à lever des
indéterminations, on peut souvent utiliser des DLs. 

\begin{Exem}~

Soit à calculer $\underset{x\rightarrow \infty}{\lim} \left(x\sin\frac{1}{x}\right)^{x^2}$

Remarquons tout d'abord que la fonction proposée est bien définie pour $x >
1/\pi$.


Commençons par passer au log. Pour tout $x>1/\pi$ :

$$ \ln \left(x\sin\frac{1}{x}\right)^{x^2} = x^2 \ln(x\sin\frac{1}{x}) $$

Effectuons le changement de variable $x=\frac{1}{t}$

$$ x^2 \ln\left(x\sin\frac{1}{x}\right) = \frac{1}{t^2}\ln \left(\frac{\sin(t)}{t}\right)$$

Un simple équivalent à l'ordre 0 ne nous permet pas de calculer la limite
de ectte expression quand $t$ tend vers 0. En effet, $\frac{\sin(t)}{t}
\underset{t\rightarrow 0}{\sim} 1$, et le log est de limite nulle. Nous
n'avons pas levé l'indétermination.

Allons un cran plus loin :

$$\frac{\sin(t)}{t}\underset{t\rightarrow 0}{=} 1 - \frac{t^2}{6}
+\text{o}(t^3)$$

et par composition de DLs

$$\ln\left(\frac{\sin(t)}{t}\right)\underset{t\rightarrow 0}{=} - \frac{t^2}{6}
+\text{o}(t^3)$$

D'où $$\underset{t\rightarrow 0}{\lim}\; \frac{1}{t^2}\ln
\left(\frac{\sin(t)}{t}\right)=-\frac{1}{6}$$

et la limite recherchée est donc

$$\underset{x\rightarrow \infty}{\lim} \left(x\sin\frac{1}{x}\right)^{x^2}
= \exp\left(-\frac{1}{6}\right)$$




\end{Exem}

\subsection{Étude de points singuliers de courbes du plan}

Commençons par quelques rappels sur les courbes du plan.


\begin{Def}[Points (bi)-réguliers et singuliers]~

\begin{itemize}
\item 
Soit $I$ un intervalle de $\R$, $f:I\ra \R^2$ un arc paramétré
$\mathcal{C}^1$, et $a\in I$.


On dit que le $f$ est \defemph{régulière} en $a$ (ou que le point $f(a)$
est régulier\footnotemark) Ssi $f'(a)\neq(0,0)$.

Dans le cas contraire, on dit que $f$ est \defemph{singulière} en $a$ (ou
que le point $f(a)$ est singulier).

\item 
Soit $I$ un intervalle de $\R$ et $f:I\ra \R^2$ un arc paramétré
$\mathcal{C}^2$.


On dit que $f$ est \defemph{birégulière}  en $a$ (ou que le point $f(a)$
  est birégulier) Ssi
  $(f'(a),f''(a))$ est libre.
\end{itemize}


\end{Def}

\footnotetext{ce vocabulaire posant un problème quand la courbe n'est pas simple\dots}


\begin{Def}[Demi-tangentes]~

Soit $f:I\ra \R^2$ un arc paramétré, et $a\in I$
\begin{itemize}


\item Supposons que pour $t$ assez proche de $a$ et $t<a$, $f(t)-f(a)$ ne
  s'annule pas et que $\vec{u}_g=\underset{t\ra a^-}{\lim}\frac{f(t)-f(a)}{||f(t)-f(a)||}$ existe. 

On appelle alors \defemph{demi-tangente à gauche} la droite passant par
$f(a)$ et dirigée par $\vec{u}_g$.

\item Supposons que pour $t$ assez proche de $a$ et $t>a$, $f(t)-f(a)$ ne
  s'annule pas et que $\vec{u}_d=\underset{t\ra
  a^+}{\lim}\frac{f(t)-f(a)}{||f(t)-f(a)||}$ existe. 

On appelle alors \defemph{demi-tangente à droite} la droite passant par
$f(a)$ et dirigée par $\vec{u}_d$.

\item Si les demi-tangentes à droite et à gauche existent et sont
  confondues (i.e. si $\vec{u}_d$ et $\vec{u}_g$ sont colinéaires), on appelle
  simplement \defemph{tangente} la droite passant par $f(a)$ et dirigée par
  l'un des vecteurs.


\end{itemize}


\end{Def}

\begin{center}
\input{images/DLdemit.pdftex_t}  

Deux demi-tangentes non colinéaires.
\end{center}



\begin{center}
\input{images/DLtan.pdftex_t}  

Deux demi-tangentes colinéaires.
\end{center}


Dans le cas d'un point régulier, nous avons la propriété suivante :

\begin{Prop}~

Soit $I$ un intervalle de $\R$ et $f:I\ra \R^2$ un arc paramétré
$\mathcal{C}^1$, et $a\in I$ tel que $f$ est régulière en $a$.

Alors $f$ admet une tangente en $a$, dirigée par $f'(a)$.

\end{Prop}

\begin{demo}~

Pour commencer, la régularité nous garantit que $f(t)-f(a)$ ne s'annule pas
pour $t\neq a$ au voisinage de $a$.

Notons $f(.)=(x(.),y(.))$.

Écrivons alors pour $t<a$ au voisinage de $a$ $$\frac{f(t)-f(a)}{||f(t)-f(a)||} = \left(\frac{x(t)-x(a)}{||f(t)-f(a)||} ,\frac{y(t)-y(a)}{||f(t)-f(a)||}\right)=\frac{1}{||\frac{f(t)-f(a)}{t-a}||} \left(\frac{x(t)-x(a)}{t-a} ,\frac{y(t)-y(a)}{t-a}\right)$$

En passant à la limite, on obtient donc

$$\underset{x\ra a^-}{\lim} \frac{f(t)-f(a)}{||f(t)-f(a)||}=\frac{(x'(a),
  y'(a))}{||f'(a)||}= \frac{f'(a)}{||f'(a)||}$$

D'où l'existence d'une demi-tangente à gauche. On obtient de même une
demi-tangente à droite dirigée par le même vecteur, et donc l'existence
d'une tangente dirigée par $f'(a)$.

\end{demo}

Nous pouvons encore raffiner dans le cadre d'un point birégulier.

\begin{Prop}~

Soit $I$ un intervalle de $\R$ et $f:I\ra \R^2$ un arc paramétré
$\mathcal{C}^2$, et $a\in I$ tel que $f$ est birégulière en $a$.

Alors $f$ admet une tangente en $a$, dirigée par $f'(a)$ et de plus, au
voisinage de $a$, la courbe est située dans le demi-plan induit par la
tangente et contenant $f''(a)$.

\end{Prop}

\begin{center}
\input{images/DLtan2.pdftex_t}  
\end{center}


\begin{demo}~

Pour la partie spécifique de cette proposition, écrivons, à l'aide de
Taylor-Young (\ref{TaylorYoung})

$$
\left\{
\begin{array}{ccc}
x(t)&\underset{t\ra
  a}{=}&x(a)+x'(a)(t-a)+x''(a)\frac{(t-a)^2}{2}+\text{o}((t-a)^2)\\
y(t)&\underset{t\ra
  a}{=}&y(a)+y'(a)(t-a)+y''(a)\frac{(t-a)^2}{2}+\text{o}((t-a)^2)\\
\end{array}
\right.
$$

Par hypothèse, $(f(a), f'(a), f''(a))$ est un repère du plan. Si l'on écrit
$f(t)=(X(t), Y(t)) $ dans ce repère, on a donc :

$$
\left\{
\begin{array}{ccc}
X(t)&\underset{t\ra
  a}{=}&(t-a)+\text{o}((t-a)^2)\\
Y(t)&\underset{t\ra
  a}{=}&\frac{(t-a)^2}{2}+\text{o}((t-a)^2)\\
\end{array}
\right.
$$

Au voisinage de $a$, $Y\geq0$, i.e. la projection de $f$ sur la droite
passant par $f(a)$ et dirigée par $f''(a)$ est donc orientée dans le même
sens que $f''(a)$.

\end{demo}


Cette propriété se généralise en fait aux points qui ne sont pas
biréguliers, de la façon suivante :


\begin{Prop}~

Soit $k\geq 2$.

Soit $I$ un intervalle de $\R$ et $f:I\ra \R^2$ un arc paramétré
$\mathcal{C}^k$, et $a\in I$.

Soit $p$, s'il existe, le plus petit entier strictement positif tel que $f^{(p)}(a)$ est
non nul. 

Alors $f$ admet une tangente en $a$, dirigée par $f^{(p)}(a)$.

Soit de plus $q$, s'il existe, le plus petit entier strictement supérieur à
$p$ tel que $f^{(p)}$ et $f^{(q)}$ ne soient pas colinéaires.

Alors, suivant les parités de $p$ et de $q$, on a l'une des 4 configurations
suivantes. 
\begin{tabular}{cc}
\input{images/DLip.pdftex_t}&\input{images/DLii.pdftex_t}\\
$p$ impair, $q$ pair & $p$ impair, $q$ impair \\
Point ordinaire & Point d'inflexion\\
\input{images/DLpi.pdftex_t}&\input{images/DLpp.pdftex_t}\\
$p$ pair, $q$ impair & $p$ pair, $q$ pair \\
Point de rebroussement de 1\textsuperscript{ère} espèce &
Point de rebroussement de 2\textsuperscript{nde} espèce \\
\end{tabular}

\end{Prop}

En particulier, dans le cas d'un point birégulier, on a un point ordinaire.


\begin{demo}~

Reprenons les notations de la propriété.

$$
\left\{
\begin{array}{ccc}
x(t)&\underset{t\ra
  a}{=}&x(a)+x^{(p)}(a)\frac{(t-a)^p}{p!}+x^{(p+1)}(a)\frac{(t-a)^{p+1}}{p+1!}+\dots +x^{(q)}(a)\frac{(t-a)^q}{q!}+\text{o}((t-a)^q)\\
y(t)&\underset{t\ra
  a}{=}&y(a)+y^{(p)}(a)\frac{(t-a)^p}{p!}+y^{(p+1)}(a)\frac{(t-a)^{p+1}}{p+1!}+\dots +y^{(q)}(a)\frac{(t-a)^q}{q!}+\text{o}((t-a)^q)\\
\end{array}
\right.
$$

Or par hypothèse, $f^{(p)}(a), f^{(p+1)}(a), \dots f^{(q-1)}(a)$ sont
colinéaires. On peut donc écrire :

$$
\left\{
\begin{array}{ccc}
x(t)&\underset{t\ra
  a}{=}&x(a)+x^{(p)}(a)\frac{(t-a)^p}{p!}(1+\lambda_{p+1}(t-a)+\dots+\lambda_{q-1}(t-a)^{q-1-p}) +x^{(q)}(a)\frac{(t-a)^q}{q!}+\text{o}((t-a)^q)\\
y(t)&\underset{t\ra
  a}{=}&y(a)+y^{(p)}(a)\frac{(t-a)^p}{p!}(1+\lambda_{p+1}(t-a)+\dots+\lambda_{q-1}(t-a)^{q-1-p}) +y^{(q)}(a)\frac{(t-a)^q}{q!}+\text{o}((t-a)^q)\\
\end{array}
\right.
$$


Par hypothèse, $(f(a), f^{(p)}(a), f^{(q)}(a))$ est un repère du plan. Si l'on écrit
$f(t)=(X(t), Y(t)) $ dans ce repère, on a donc :

$$
\left\{
\begin{array}{ccc}
X(t)&\underset{t\ra
  a}{=}&\frac{(t-a)^p}{p!}+\text{o}((t-a)^p)\\
Y(t)&\underset{t\ra
  a}{=}&\frac{(t-a)^q}{q!}+\text{o}((t-a)^q)\\
\end{array}
\right.
$$

L'étude des signes de $X(t)$ et $Y(t)$ au voisinage de $a$ suivant les
parités de $p$ et $q$ donnent alors le résultat.
\end{demo}


\begin{Ex}~

\'Etudier la courbe param\'etr\'ee d\'efinie pour tout $r\in R$ par

$$
\left\{ \begin{array}{rcl}
x(t)&=&\cos^3(t)\\
y(t)&=&\sin^3(t)\\
\end{array}\right.
$$

\end{Ex}

\begin{demo}~

  \begin{description}
  \item[R\'eduction de l'intervalle d'\'etude:]~

    \begin{itemize}
    \item     $x$ et $y$ sont $2\pi$ p\'eriodiques. On peut donc
      restreindre l'intervalle d'\'etude \`a $[-\pi,\pi]$.
    \item $x$ est paire et $y$ est impaire. Par sym\'etrie par rapport au
      premier axe, on peut restreindre l'intervalle d'\'etude \`a $[0,\pi]$.
    \item On a $\forall t\in [0,\pi]$, $x(\pi-t)=-x(t)$ et
      $y(\pi-t)=y(t)$. Par sym\'etrie par rapport au second axe, on peut
      restreindre l'intervalle d'\'etude \`a $[0,\pi/2]$.
    \item  On a $\forall t\in [0,\pi/2]$, $x(\pi/2-t)=y(t)$ et $\forall
      t\in [0,\pi/2]$, $y(\pi/2-t)=x(t)$. Par sym\'etrie par rapport \`a la
      premi\`ere bissectrice des axes, on peut
      restreindre l'intervalle d'\'etude \`a $[0,\pi/4]$.

    \end{itemize}

  \item[\'Etude des variations:]~

    Sur $[0, \pi/4]$, $x$ est d\'ecroissante et $y$ croissante. On a donc
    le tableau de variation suivant :
    
$$
    \begin{array}{|c|ccc|}
      \hline 
      & 0 & & \pi/4\\
      \hline 
      x(t) & 1 & & 0\\
      \hline 
      y(t) & 0 & & 1\\
\hline
    \end{array}
 $$
  
\item[\'Etude des points singuliers:]~
 
La courbe admet un point singulier en $t$ Ssi 

$$
\left\{ \begin{array}{lcccr}
x'(t)&=&-3\sin(t) \cos^2(t)&=&0\\
y'(t)&=&3\cos(t)\sin^2(t)&=&0\\
\end{array}\right. 
\Leftrightarrow \cos(t)=0\; \text{ou}\; \sin(t)=0
\Leftrightarrow t=0\left[\frac{\pi}{2}\right]
$$

\'Etant donn\'e notre intervalle d'\'etude, nous allons regarder ce qui se
passe en 0. Effectuons des DLs de $x$ et $y$ en 0 :

$$
\left\{ \begin{array}{rcl}
x(t)&\underset{t\ra 0}{=}&1-\frac{3x^2}{2}+\frac{7x^4}{8}+\text{o}(t^5)\\
y(t)&\underset{t\ra 0}{=}&x^3-\frac{x^5}{2}+\text{o}(t^5)\\
\end{array}\right.
$$

\'Ecrivons les d\'eriv\'ees vectorielles successives :

$$f'(t)=(0,0)\;\;\;f^{(2)}(t)=(3,0)\;\;\;f^{(3)}(t)=(0,6)$$

Avec les notations du th\'eor\`eme pr\'ec\'edent, $p=2$ et $q=3$.

La courbe admet donc une tangente horizontale en $0$, et un point de
rebroussement de seconde esp\`ece en ce point (ce qui assez rassurant au vu
des symétries de la courbe.)

  \end{description}

Nous obtenons donc la courbe que voici, que l'on appelle \emph{astroïde} :

\begin{center}
\includegraphics[width=8cm]{images/astroide.png}

\end{center}

\end{demo}


\section{Développements asymptotiques}

Toute la théorie que nous avons présentée ici permet d'effectuer des
approximations de fonctions au voisinage d'un point réel par des
polynômes. Or, il est parfois important de trouver des approximations de
fonctions au voisinage de $+\infty$, ce qui permet en particulier de
trouver des asymptotes.

Tout comme dans l'exercice précédent, on va se ramener à un calcul de DL en
0 de la façon suivante :

\smallskip
Soit $f:[m,+\infty[ \rightarrow \R$, dont nous voulons trouver une
    approximation en $+\infty$.


\begin{itemize}


\item On pose $g:t \rightarrow f(1/t)$ pour $t<1/m$

\item On étudie la fonction $g$ au voisinage de 0. Supposons par exemple
  que $g$ admette un $\DL_n(0)$ pour un certain $n\inN$ :
  $g(t)\underset{t\rightarrow 0^+}{=}P(t)+\text{o}(t)^n$

\item En composant à droite, on obtient   $f(x)\underset{x\rightarrow +\infty}{=}P(1/x)+\text{o}(1/x)^n$

\end{itemize}

et on a donc approché $f$ par une fraction rationnelle au voisinage de
$+\infty$.

\begin{Exems}~

\begin{itemize}

\item Formons un développement asymptotique de $x\rightarrow \sqrt{x(x+1)}$ à la
  précision $\frac{1}{x}$. 


Pour $t>0$, posons $$g:t\rightarrow \sqrt{\frac{1+t}{t^2}}=\frac{1}{t}\sqrt{1+t}$$

On a donc $$g(t)\underset{t\rightarrow 0^+}{=} \frac{1}{t} (1+\frac{t}{2}-\frac{t^2}{8}+\text{o}(t^2))=\frac{1}{t}+\frac{1}{2}-\frac{t}{8}+\text{o}(t)$$

En revenant à $f$,

$$f(x)\underset{x\rightarrow +\infty}{=} x+\frac{1}{2}-\frac{1}{8x}+\text{o}\left(\frac{1}{x}\right)$$

Ce qui nous dit $f$ admet la droite d'équation $y=x+1/2$ comme asymptote en
$+\infty$, et que de plus, elles est en dessous de cette droite.

\item Tentons de trouver une approximation en $+\infty$ de $f:x\ra
  \sqrt{x+1}\ln(x) - \sqrt{x}\ln(1+x)$.

Chacun des termes est équivalent à $x\ra \sqrt{x}\ln(x)$ en
$+\infty$. Commençons donc par factoriser par cette quantité pour y voir
plus clair. Pour $x$ assez grand, on a 

$$f(x)=\sqrt{x}\ln(x)\left(\sqrt{\frac{x+1}{x}}-\frac{\ln(x+1)}{\ln(x)}\right) $$

Effectuons le changement de variable $g(t)=f(1/t)$, et calculons un peu.

\begin{align*}
g(t)
&=-\sqrt{1/t}\ln(t)\left(\sqrt{t+1}-\frac{\ln(1/t+1)}{\ln(1/t)}\right)\\
&= -\sqrt{1/t}\ln(t)\left(\sqrt{t+1}-1+\frac{\ln(t+1)}{-\ln(t)}\right) \\
\end{align*}

Nous connaissons les DLs à tout ordre en 0 de $\sqrt{t+1}$ et de
$\ln(t+1)$. Pas moyen cependant de se débarasser du $\ln(t)$. Nous
pouvons donc écrire :

$$\sqrt{t+1}-1\underset{t\rightarrow 0}{=}
\frac{t}{2}-\frac{t^2}{8}+ \dots$$ 

et 

$$
\frac{\ln(t+1)}{-\ln(t)}\underset{t\rightarrow 0}{=}-\frac{t}{ln(t)}+\frac{t^2}{2ln(t)}+\dots
$$

Ce second développement fait appel à une échelle différente de l'échelle à
laquelle nous sommes habitués avec les DLs (1, $t$, $t^2$, \dots). Ici,
nous avons des $\frac{t}{\ln(t)}$, $\frac{t^2}{\ln(t)}$,
$\frac{t^3}{\ln(t)}$\dots

On peut montrer, à l'aide des comparaisons des vitesses de convergence de
fonctions classiques, que l'on a les imbrications suivantes :

$\dots \text{o}(\frac{t^2}{\ln(t)})\underset{t\ra 0}{\subset} \text{o}(t^2)\underset{t\ra 0}{\subset} \text{o}(\frac{t}{\ln(t)})\underset{t\ra 0}{\subset} \text{o}(t)$

Allons par exemple jusqu'à la précision $t^2$

\begin{align*}
g(t)
&=-\sqrt{1/t}\ln(t)\left(\frac{t}{2}-\frac{t^2}{8}-\frac{t}{\ln(t)}+\text{o}(t^2)\right)\\
&= -\frac{\sqrt{t}\ln(t)}{2}+\frac{t\sqrt{t}\ln(t)}{8}+\sqrt{t}+\text{o}(\ln(t)t\sqrt{t})
\end{align*}


Soit, en revenant à $f$, 

\begin{align*}
f(x)
&= \frac{\ln(x)}{2\sqrt{x}}+\frac{1}{\sqrt{x}}-\frac{\ln(x)}{8x\sqrt{x}}+\text{o}\left(\frac{\ln(x)}{x\sqrt{x}}\right)
\end{align*}


\end{itemize}

Le développement obtenu fait ici intervenir autre chose que des fractions
rationnelles (et cette fonction n'est en fait équivalente à aucune fraction
rationnelle en $+\infty$, comme on peut aisément s'en convaincre).

\end{Exems}


\newpage
\section*{Correction des exercices}

\addcontentsline{toc}{section}{Correction des exercices}

\Closesolutionfile{test}
   \Readsolutionfile{test}
   \Closesolutionfile{testtwo}
   \Readsolutionfile{testtwo}


%%%%%%%%%%%TODO : exemples avec des propositions du type x>4

\end{document}