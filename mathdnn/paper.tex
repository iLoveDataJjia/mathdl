\documentclass[11pt,en]{elegantpaper}

% Docs
\title{MathDNN - A deep mathematical understanding of DNNs}
\author{James JIANG \\ Data Engineer / Scientist \\ France \and Alex JIANG \\ Preparatory class for the Grandes Écoles \\ France}
\institute{\href{https://github.com/iLoveDataJjia}{iLoveDataJjia Github}}

\version{0.00}
\date{\today}

% Packages
\usepackage{array}
\usepackage{amsthm}
\usepackage{amsmath}
\usepackage{fourier}
\usepackage{accents}
\newtheorem{convention}{Convention}
\newtheorem{notation}{Notation}

% Custom commands
\newcommand\norm[1]{\left\lVert#1\right\rVert}

% Start build
\begin{document}

\maketitle

\begin{abstract}
  Frameworks such as TensorFlow or PyTorch make deep learning developments easy.
  They have made this field wide spread for every enthusiast.
  Implementations only needs an instinctive understanding of deep learning.
  The proper math aspect is little by little forgotten.
  Topology, Normalized vector space, Limit plus continuity, Taylor series expansion, Matrix,
  Finite dimensional linear algebra and Linear application matrix theories are supposed known.
  The objective is to do a collection of the important propositions explaining dense neural
  network (DNN) theories. These propositions will be mathematically proven.
  The subject used as reference is a multi-class classification problem with
  – dense layers, andactivation layers, Categorical cross-entropy loss and Stochastic
  gradient descent optimizer. But all the elements below can be easily re-used or re-defined
  to cover regressions.
  \keywords{Dense neural network, Equation, Asumption, Proof}
\end{abstract}

\section{Fundamentals and notations}

\subsection{Matrices}

% Matrix notation
\begin{notation}
  Let $a_{i,j} \in \mathbb{R}$ for $i \in \llbracket 1,n \rrbracket$ and $j \in \llbracket 1,m \rrbracket$.
  Then a real matrix of dimension $n * m$ will noted as \begin{equation*}
    A = \begin{bmatrix}
      a_{1,1} & a_{1,2} & \cdots & a_{1,m} \\
      a_{2,1} & a_{2,2} & \cdots & a_{2,m} \\
      \vdots & \vdots & \ddots & \vdots \\
      a_{n,1} & a_{n,2} & \cdots & a_{n,m}
    \end{bmatrix}
  \end{equation*}

  The following notations are also considered \begin{gather*}
    \forall i \in \llbracket 1,n \rrbracket, \forall j \in \llbracket 1,m \rrbracket, A_{i,j} = a_{i,j} \\
    \forall j \in \llbracket 1,m \rrbracket, A_{:,j} = \begin{bmatrix}
      a_{1,j} \\
      \vdots \\
      a_{n,j}
    \end{bmatrix} \\
    \forall i \in \llbracket 1,n \rrbracket, j \in \llbracket 1,m \rrbracket, A_{i,j} = \begin{bmatrix}
      a_{i,1} & \cdots & a_{i,n}
    \end{bmatrix}
  \end{gather*}

  The notation $\mathcal{M}_{n,m}$ means the matrix set of dimension $n \times m$ with coefficients in $\mathbb{R}$. \par
  The notation $\mathcal{M}_{n,m}(E)$ means the matrix set of dimension $n \times m$ with coefficients in $E \subseteq \mathbb{R}$.
\end{notation}

% Vector convention to matrix row
\begin{convention}
  A vector is a matrix with only one row. Thus, the real vector set $\mathbb{R}^n$ is equivalent to $\mathcal{M}_{1,n}$.
\end{convention}

% Matrix product
\begin{notation}
  Let $A \in \mathcal{M}_{n,m}$ and $B \in \mathcal{M}_{m,p}$. Let the product noted $A * B$ be \begin{gather*}
    C = A * B
  \end{gather*}
  where $C \in \mathcal{M}_{n,p}$ with \begin{gather*}
    \forall i \in \llbracket 1,n \rrbracket, \forall j \in \llbracket 1,p \rrbracket, C_{i,j} = \sum_{k=1}^n A_{i,k} * B_{k,j}
  \end{gather*}

\end{notation}

% Matrix transpose
\begin{notation}
  The matrix transpose operation will be noted as $A^T$.
\end{notation}

% Norm
\begin{notation}
  Let $a \in \mathbb{R}^n$. The eucliean norm on $\mathbb{R}^n$ will be noted as $\norm a _n$.
  \begin{gather*}
    \norm a _n = \sqrt{a * a^T}
  \end{gather*}
\end{notation}

\subsection{Differential calculus}

% Set not empty
\begin{convention}
  All sets considered are not empty.
\end{convention}

% Base notations
\begin{notation}
  Let $E \subseteq \mathbb{R}^n$ and $F \subseteq \mathbb{R}^m$. \par
  The notation $\mathring{E}$ means the set with only the interior point of $E$. \par
  The notation $f : E \longrightarrow F$ means the application from $E$ to $F$. \par
  The notation $\mathcal{C}(E,F)$ means the set of continuous applications from $E$ to $F$. \par
  The notation $\mathcal{L}(E,F)$ means the set of linear applications from $E$ to $F$.
\end{notation}

% Differentiable definition
\begin{definition}
  Let $E \subseteq \mathbb{R}^n$ and $F \subseteq \mathbb{R}^m$.
  Then $f$ differentiable on $E$ is equivalent to \begin{equation}\label{eq:differentiable}
    \begin{gathered}
      \forall a \in \mathring{E}, \exists \frac{\partial f}{\partial \cdot}(a) \in \mathcal{L}(\mathbb{R}^n,\mathbb{R}^m), \\
      \forall h \in \mathbb{R}^n, f(a+h) = f(a) + \frac{\partial f}{\partial h}(a) + \underset{h \to 0}o(\norm h _n)
    \end{gathered}
  \end{equation}

  $\frac{\partial f}{\partial \cdot}(a)$ is named differential of $f$ on $a$. \par
  The notation $\mathcal{D}(E,F)$ means the set of differentiable applications from $E$ to $F$.
\end{definition}

% Differentiable properties
\begin{proposition}\label{prop:differential_unique}
  {\normalfont Let $E \subseteq \mathbb{R}^n$, $F \subseteq \mathbb{R}^m$ and $f \in \mathcal{D}(E,F)$.
  Then $\frac{\partial f}{\partial \cdot}(a)$ is unique and $\mathcal{D}(E,F) \subset \mathcal{C}(E,F)$.}
\end{proposition}

\begin{proof}
  Suppose $\phi_1$ and $\phi_2$ two differentiales of $f$ on $a$. \begin{equation*}
    \begin{gathered}
      \forall h \in \mathbb{R}^n, \phi_2(h) - \phi_1(h) \underset{(\ref{eq:differentiable})} = \underset{h \to 0}o(\norm h _n)
    \end{gathered}
  \end{equation*}
  \begin{equation*}
    \begin{split}
      & \underset{def} \implies \forall \epsilon > 0, \exists \eta > 0, \forall h \in \mathbb{R}^n, (\norm h _n \leq \eta \Rightarrow \norm{\phi_2(h) - \phi_1(h)}_m \leq 2 * \norm h _n * \epsilon) \\
      & \underset{\phi_2 - \phi_1 \in \mathcal{L}(\mathbb{R}^n,\mathbb{R}^m)} \implies \forall \epsilon > 0, \forall h \in \mathbb{R}^n, \norm{\phi_2(h) - \phi_1(h)}_m \leq 2 * \norm h _n * \epsilon \\
      & \underset{\epsilon \to 0} \implies \forall h \in \mathbb{R}^n, \phi_2(h) = \phi_1(h)
    \end{split}
  \end{equation*}
  
  Let $f \in \mathcal{D}(E,F).$ and $a \in \mathring{E}$. \begin{equation*}
    \begin{split}
      \frac{\partial f}{\partial \cdot}(a) \in \mathcal{L}(\mathbb{R}^n,\mathbb{R}^m) & \implies \frac{\partial f}{\partial \cdot}(a) \in \mathcal{C}(\mathbb{R}^n,\mathbb{R}^m), \frac{\partial f}{\partial 0_{\mathbb{R}^n}}(a) = 0_{\mathbb{R}^m} \label{} \\
      & \underset{(\ref{eq:differentiable})} \implies f(a + h) \underset{h \to 0} \to f(a)
    \end{split}
  \end{equation*}
\end{proof}

% Jacobian definition
\begin{definition}
  Let $E \subseteq \mathbb{R}^n$, $F \subseteq \mathbb{R}^m$ and $f = (f_1 \ldots f_m) \in \mathcal{D}(E,F)$.
  Then $f_i$ is differentiable on $E$ for all $i \in \llbracket 1,m \rrbracket$. The jacobian is defined as \begin{equation}
    \begin{array}{llll}
      \mathcal{J}_f & : & \mathring{E} & \longrightarrow \mathcal{M}_{m,n} \\
        &   & a & \longmapsto \begin{bmatrix}
        \frac{\partial f}{\partial e_1}(a) & \cdots & \frac{\partial f}{\partial e_n}(a)
      \end{bmatrix} = \begin{bmatrix}
        \frac{\partial f_1}{\partial e_1}(a) & \cdots & \frac{\partial f_1}{\partial e_n}(a) \\
        \vdots & \ddots & \vdots \\
        \frac{\partial f_m}{\partial e_1}(a) & \cdots & \frac{\partial f_m}{\partial e_n}(a) \\
      \end{bmatrix}
    \end{array}
  \end{equation}

  $(e_i)_{i \in \llbracket 1,n \rrbracket}$ means the matrices $e_i = \begin{bmatrix}
    0 & \cdots & \underset{\text{at index 0}} 1 & \cdots & 0
  \end{bmatrix}$ corresponding to $\mathbb{R}^n$ standard basis. \par
  $\frac{\partial f}{\partial e_i}$ is named the partial derivative of $f$ according the $i^{th}$ variable. \par
  The jacobian is also named gradient when $m=1$ and is noted as $\nabla_f = \mathcal{J}_f$.
\end{definition}

\begin{proof}
  Suppose $f = (f_1 \ldots f_m) \in \mathcal{D}(E,F)$. Let $i \in \llbracket 1,m \rrbracket$, $a \in \mathring{E}$ and $h \in \mathbb{R}^n$.
  \begin{equation*}
    \begin{split}
      f_i (a + h) & = f(a + h)_i \\
      & = f(a)_i + \frac{\partial f}{\partial h}(a)_i + \underset{h \to 0}o(\norm h _n)_i \\
      & = f_i(a) + \frac{\partial f}{\partial h}(a)_i + \underset{h \to 0}o(\norm h _n)_i
    \end{split}
  \end{equation*}
  \begin{equation*}
    \frac{\partial f}{\partial \cdot}(a)_i \in \mathcal{L}(\mathbb{R}^n,\mathbb{R}) \underset{prop\ref{prop:differential_unique}}\implies \frac{\partial f_i}{\partial h}(a) = \frac{\partial f}{\partial h}(a)_i
  \end{equation*}
\end{proof}

% Jacobian definition corollary
\begin{corollary}
  {\normalfont Let $E \subseteq \mathbb{R}^n$, $F \subseteq \mathbb{R}^m$ and $f \in \mathcal{D}(E,F)$.
  The jacobian of $f$ on $a \in \mathring{E}$ fixed is the canonical associated matrix to the differential of $f$ on $a$.} \par
  \normalfont{\textbf{Notes:} It means a function differentiability can also be shown by exhibing its jacobian.} \par
\end{corollary}

\begin{proof}
  Let $a \in \mathring{E}$.
  $\frac{\partial f}{\partial \cdot}(a) \in \mathcal{L}(\mathbb{R}^n,\mathbb{R}^m)$ and any linear application in finite dimension
  with values in $\mathbb{R}$ has an unique associated matrix in the standard basis called canonical associated matrix. \par
\end{proof}

% Function composition notation
\begin{notation}
  Let $f : E \longrightarrow F$ and $g : F \longrightarrow G$.
  Then the notation $g \circ f$ means the application \begin{equation*}
    \begin{array}{llll}
      g \circ f & : & E & \longrightarrow G \\
        &   & x & \longmapsto g(f(x))
    \end{array}
  \end{equation*} \par

  Let $f_i : E_i \longrightarrow E_{i+1}$ for $i \in \llbracket 1,n \rrbracket$.
  Then the notation $\underset{i=1}{\overset{n}\circ} f_i$ means the application \begin{equation*}
    \begin{array}{llll}
      \underset{i=1}{\overset{n}\circ} f_i & : & E_1 & \longrightarrow E_{n+1} \\
        &   & x & \longmapsto f_n( \ldots f_2(f_1(x)))
    \end{array}
  \end{equation*}
\end{notation}

% Chain rule
\begin{theorem}
  {\normalfont Let $E \subseteq \mathbb{R}^n$, $F \subseteq \mathbb{R}^m$, $G \subseteq \mathbb{R}^p$, $f \in \mathcal{D}(E,F)$ and $g \in \mathcal{D}(E,F)$.
  Then $g \circ f \in \mathcal{D}(E,G)$ and} \begin{equation}
    \begin{array}{llll}
      \mathcal{J}_{g \circ f} & : & \mathring{E} & \longrightarrow G \\
      &   & a & \longmapsto \mathcal{J}_{g}(f(a)) * \mathcal{J}_{f}(a)
    \end{array}
  \end{equation}
\end{theorem}

\begin{proof}
  Let $E \subseteq \mathbb{R}^n$, $F \subseteq \mathbb{R}^m$, $G \subseteq \mathbb{R}^p$, $f \in \mathcal{D}(E,F)$ and $g \in \mathcal{D}(F,G)$.
  Let $a \in \mathring{E}$ and $h \in \mathbb{R}^n$.
  \begin{equation*}
    \begin{split}
      (g \circ f)(a+h) & = g(f(a) + \frac{\partial f}{\partial h}(a) + \underset{h \to 0}o(\norm h _n)) \\
      & = g(f(a)) + \frac{\partial g}{\partial (\frac{\partial f}{\partial h}(a) + \underset{h \to 0}o(\norm h _n))}(f(a))
        + \underset{h \to 0}o(\norm{\frac{\partial f}{\partial h}(a) + \underset{h \to 0}o(\norm h _n)}_n) \\
      & \underset{\frac{\partial f}{\partial \cdot}(a) \in \mathcal{C}(\mathbb{R}^n,\mathbb{R}^m),\frac{\partial f}{\partial 0_{\mathbb{R}^n}}(a) = 0_{\mathbb{R}^m}}
        = g(f(a)) + \frac{\partial g}{\partial (\frac{\partial f}{\partial h}(a) + \underset{h \to 0}o(\norm h _n))}(f(a)) + \underset{h \to 0}o(\norm h _n) \\
      & \underset{\frac{\partial g}{\partial \cdot}(a) \in \mathcal{L}(\mathbb{R}^m,\mathbb{R}^p)}
        = g(f(a)) + \frac{\partial g}{\partial (\frac{\partial f}{\partial h}(a))}(f(a)) + \frac{\partial g}{\partial (\underset{h \to 0}o(\norm h _n))}(f(a)) + \underset{h \to 0}o(\norm h _n) \\
      & \underset{\frac{\partial g}{\partial \cdot}(a) \in \mathcal{C}(\mathbb{R}^m,\mathbb{R}^p),\frac{\partial g}{\partial 0_{\mathbb{R}^m}}(a) = 0_{\mathbb{R}^p}}
        = g(f(a)) + \frac{\partial g}{\partial (\frac{\partial f}{\partial h}(a))}(f(a)) + \underset{h \to 0}o(\norm h _n)
    \end{split}
  \end{equation*}
  \begin{equation*}
    \begin{split}
      \frac{\partial g}{\partial (\frac{\partial f}{\partial \cdot}(a))}(f(a)) = \frac{\partial g}{\partial \cdot}(f(a)) \circ \frac{\partial f}{\partial \cdot}(a) \in \mathcal{L}(\mathbb{R}^n, \mathbb{R}^p)
      & \underset{prop\ref{prop:differential_unique}}\implies g \circ f \in \mathcal{D}(\mathbb{R}^n, \mathbb{R}^p), \frac{\partial (g \circ f)}{\partial \cdot}(a) = \frac{\partial g}{\partial \cdot}(f(a)) \circ \frac{\partial f}{\partial \cdot}(a) \\
      & \underset{mat}\implies \mathcal{J}_{g \circ f}(a) = \mathcal{J}_{g}(f(a)) * \mathcal{J}_{f}(a)
    \end{split}
  \end{equation*} \par

  \textbf{Note:} $mat$ indicates in canonical associated matrix way. \par
\end{proof}

\subsection{Others}
 
% Indicator function
\begin{notation}
  Let $E \subseteq \mathbb{R}^n$. The notation $\mathbb{1}_E$ means the $E$ indicator function on $\mathbb{R}^n$.
  \begin{equation*}
    \begin{array}{llll}
      \mathbb{1}_E & : & E & \longrightarrow \{1,0\}^n \\
      &   & x & \longmapsto \begin{array}{ll}
        1 & x \in E \\
        0 & x \notin E
      \end{array}
    \end{array}
  \end{equation*}
\end{notation}

% Fixed variable function
\begin{notation}
  Let $f$ an application with $n$ inputs and $m$ outputs.
  \begin{equation*}
    \begin{array}{llll}
      f & : & E_1 \times \ldots \times E_n & \longrightarrow F_1 \times \ldots \times F_m \\
      &   & (x_1, \ldots, x_n) & \longmapsto f(x_1, \ldots, x_n)
    \end{array}
  \end{equation*} \par

  Let $k \in \llbracket 1,n \rrbracket$. The notation $f(x_1, \ldots, x_{k-1}, \cdots, x_{k+1}, \ldots, x_n)$ means \begin{equation*}
    \begin{array}{llll}
      f(x_1, \ldots, x_{k-1}, \cdots, x_{k+1}, \ldots, x_n) & : & E_k & \longrightarrow F_1 \times \ldots \times F_m \\
      &   & x_k & \longmapsto f(x_1, \ldots, x_{k-1}, x_k, x_{k+1}, \ldots, x_n)
    \end{array}
  \end{equation*}
\end{notation}

\section{Activation functions}

% ReLU
\begin{definition}
  Let $F_{act} \in \mathcal{D}(\mathbb{R}^m,\mathbb{R}^m)$. The $F_{act}$ is an activation function.
\end{definition}

\begin{definition}
  Let the application $ReLU$ 
\end{definition}

\end{document}